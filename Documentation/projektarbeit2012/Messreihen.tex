\label{subCap:Messreihen}

Die Messreihen und Tests wurden mit einem Notebook aufgenommen, der Folgende
Hardwarekonfigurationen besitzt.

\textit{
	LENOVO Thinkpad t410 Modell 2522-w29 \newline
	\textbf{Prozessor}: Intel Core i5 520M (2 Kerne mit Hyperthreading $\stackrel{\wedge}=$ 4 Threads)\\
	\textbf{RAM}: 4 GB DDR3	
	}

Da eine ähnliche Hardware bei einer Verwendung des Protokolls auf dem Mars
weniger zu erwarten ist, sind die nachfolgenden Ergebnisse bezüglich ihrer
absoluten Werte nicht wirklich aussagekräftig. Für unsere Betrachtungen waren
jedoch mehr die Beziehungen und Entwicklungen der einzelnen Werte zueinander
wichtig.

Das Protokoll wurde immer grundlegend mit zwei verschiedenen
Compilereinstellungen getestet. Zum Einen Geschwindigkeits- (\glqq O$2$ \grqq)
und zum Anderen Quellcodegrößen-Optimiert (\glqq Os \grqq). Dabei wurde
untersucht, wie lange das Protokoll für Texte verschiedener Größe benötigt, die
keine relevanten Bereiche besitzen. In diesem Fall ist keine Vorpriorisierung
notwendig und das Protokoll kann ohne großen Aufwand die Datei splitten und die
Datenblock nacheinander verschicken. Zum Vergleich dazu ist eine Messreihen mit
unterschiedlich vielen relevanten Bereichen für einen $5.000.000$ Byte Text
aufgenommen worden. Hier ist zu erwarten, dass die Laufzeit gegenüber ersterer
Betrachtung ansteigt. 

Die eben vorgestellten Messungen wurden auf zwei verschiedenen Wegen
aufgenommen. Diese unterscheiden sich dahingehend, dass bei einem vor jedem
neuen aufzunehmenden Messwert der RAM gelöscht wurde. Für den zweiten Fall
wurden die Daten alle hintereinander verschickt um eine realistische Simulation
darzustellen zu können. 

Für einen aussagekräftigen Wert und um etwaige Messungenauigkeiten zu
kompensieren, wurde die Messung für jede Einstellung $10$ Mal wiederholt und der
Mittelwert gebildet. Die Messreihen sind aufgrund der besseren Übersicht im
Anhang zu finden (\ref{a:messtabellen}).