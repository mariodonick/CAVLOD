Der Protocol Stack bzw. Protokollstapel bezeichnet die Verschachtelung der
einzelnen Schichten die zur Erf{\"u}llung einer Protokollkommunikation
notwendig sind. Dabei werden die Daten der vorherigen schicht genommen und zur
Weiterverarbeitung der nachfolgenden Schicht zugef{\"u}hrt. Diese Stapelprinzip
gibt dem Protocol Stack seinen Namen. Der Stack {\"u}bernimmt dabei
s{\"a}mtliche Koordination und Verarbeitung der Datenstr{\"o}me (z.B.
Segmentierungen, Adressierung etc.).
Je nach Anforderung an die Kommunikation muss der Stack angepasst werden. Die
eingehenden Daten werden dann im top down Verfahren durch den Stack verpackt und
einer Sendqueue zugef{\"u}hrt nach dem Versenden werden sie auf der
Empfangsseite im bottom up Verfahren wieder ausgepackt und k{\"o}nnen dann ausgewertet werden.