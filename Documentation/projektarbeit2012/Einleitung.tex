Seit {\"u}ber f{\"u}nfzig Jahren entwickeln Forscher Strategien zur Erkundung
des Weltraums. Dabei hat sich seit der Entsendung des ersten k{\"u}nstlichen
Satelliten namens Sputnik und der ersten Mondlandung viel getan.
Zahlreiche Sonden oder Landefahrzeuge wurden seither zu diversen Zielen unseres
Sonnensystems entsandt um neue Erkenntnisse {\"u}ber unser Universum und dessen
Entstehung zu sammeln. Ein aktuelles Beispiel stellt der Mars Rover Curiosity
dar, welcher j{\"u}ngst im Gale Krater auf dem Mars landete. Der Rover
verf{\"u}gt {\"u}ber zahlreiche Messinstrumente, darunter Kameras, Sensoren zur
Messung kosmischer Strahlung, Spektrografen, meteorologische Messinstrumente und
vieles mehr. Der Rover selbst ist mit einem Gewicht von $900$ kg und der
Gr{\"o}{\ss}e eines kompakten Kleinwagens das bisher mit Abstand gr{\"o}{\ss}te
vom Menschen geschaffene Objekt das je die Marsoberfl{\"a}che erreichte.
Dieser stellt ein mobiles Labor dar, das neben den Sensoren und
Messinstrumenten auch Greifarme und R{\"a}der zur Fortbewegung und Interaktion
aufweist. Alle Messinstrumente und s{\"a}mtliche Peripherie des Rovers werden dabei von der
Erde aus gesteuert. Mit steigender Komplexit{\"a}t der mobilen Labore
steigt nat{\"u}rlich auch die Menge an Daten die es zu {\"u}bertragen gilt. Da
die komplette Steuerung und Auswertung auf der Erde stattfindet, gilt es sowohl
Steuerdaten von der Erde an den Rover zu senden, als auch Messdaten, Bilder und
{\"a}hnliches vom Rover an die Erde zu {\"u}bermitteln. Dabei unterliegt die
Kommunikation einigen entscheidenden Restriktionen. Die wohl Wichtigste ist die
Lichtgeschwindigkeit, welche sich im Vakuum mit $299 792 458$ m/s ausbreitet.
Die Ausbreitung der entsandten Signale kann daher niemals schneller erfolgen. Da die
Entfernung zum Mars je nach Konstellation variiert, ist es im Folgenden
hilfreich von der mittleren minimalen Entfernung zur Erde auszugehen. Diese
betr{\"a}gt $0,52$ AE oder aber ca. $77,8$ Millionen Kilometer (Schwankung des
$[Erde - Mars]$ Abstandes je nach Konstellation: $0,372 - 2,683$ AE, d.h. bei
Opposition $[Sonne - Erde - Mars]$ zwischen ca. $55 - 105$ Millionen Kilometern
und bei Konjunktion $[Mars - Sonne - Erde]$ ca. $400$ Millionen Kilometer).
Ausgehend von dieser Entfernung, bei einer Ausbreitungsgeschwindigkeit der
Signale von ca. $300.000$ km/s, erh{\"a}lt man eine Dauer von ca. $4,3$ Minuten
pro {\"U}bertragungsrichtung. Daraus folgt, dass die Antwort auf eine an den
Mars gesandte Anfrage bei diesem Szenario im Idealfall nach $8,6$
Minuten die Erde erreicht (im Fall der maximalen Entfernung zwischen Erde und
Mars erst nach $44$ Minuten, wobei in dieser Konstellation eine direkte
Funkverbindung nahezu unm{\"o}glich ist).
Da der Mars jedoch genau wie die Erde eine Rotation um seine eigene Achse
ausf{\"u}hrt und die Kommunikation {\"u}ber eine marsstation{\"a}re
Sendevorrichtung (Satellit) erfolgt, wird das Zeitfenster der {\"U}bertragung
weiter beschr{\"a}nkt. Diese physikalisch begr{\"u}ndeten Latenzen sind ein
wichtiger Faktor den es in der interplanetaren Kommunikation zu
ber{\"u}cksichtigen gilt. Die hieraus abzuleitende Erkenntnis ist, dass eine
Kommunikation beispielsweise zwischen Erde und Mars nicht mit einer heutigen
Irdischen Kommunikation zu vergleichen ist. Zwar ist das Modell der
verz{\"o}gerten Kommunikation dem Menschen keineswegs unbekannt.
Hier sei Exemplarisch auf den vor der Erfindung der Telegraphie
ausschlie{\ss}lich betriebenen Briefverkehr verwiesen. Jedoch ist dieses
Ph{\"a}nomen in der heutigen vernetzten Generation nahezu in Vergessenheit
geraten. Die irdische Kommunikation im $21$. Jahrhundert setzt auf geringe
Latenzen und kann somit idealisiert als verz{\"o}gerungsfrei betrachtet werden.
Doch eine interplanetare Kommunikation wird nicht allein durch die
unvermeidlichen Latenzen beeinflusst. Auch die Bandbreite unserer
{\"U}bertragung unterliegt auf der Erde abgewandten Seite
einer Beschr{\"a}nkung. Am Beispiel der Mars Rover sei hier auf die genutzten
Solarpanele bzw. eingesetzten Batterien (zumeist Radionuklidbatterien)
verwiesen, welche nur eine begrenzte Energieversorgung gew{\"a}hrleisten und
somit keine nahezu unbegrenzte Sendeleistung erm{\"o}glichen, wie es auf der
Erde der Fall ist.
Ein weiterer Punkt ist die f{\"u}r den Rover eingesetzte Technologie. Hierbei 
sind unter anderem Faktoren wie Gr{\"o}{\ss}e und Gewicht von Bedeutung, da 
diese die Kosten der Unternehmung stark beeinflussen. So kostet beispielsweise 
der Start einer Ariane $5$ Rakete zwischen $125$ und $155$ Millionen Euro bei 
einer Nutzlast von $23.000$ kg im LEO (Low Earth Orbit: $200-2.000$ km). Hieraus 
ergibt sich f{\"u}r dieses Beispiel ein Preis von ca. $7.000$ Euro pro kg 
Nutzlast. An diesem Beispiel zeigt sich die Wichtigkeit einer genauen
Kalkulation des Gewichtes der Rover. Diese Restriktion wirkt sich direkt auf die
verbaute Hardware des Rovers aus und somit auch indirekt auf Ressourcen wie z.B.
die Gr{\"o}{\ss}e des verbauten Speichers zum ablegen lokaler Daten.
Zusammenfassend kann anhand der Beispiele festgestellt werden, dass Latenz, 
Bandbreite sowie lokaler Speicher wichtige Restriktionen sind, denen die 
interplanetare Kommunikation unterliegt. Diese Voraussetzungen verlangen 
eine besondere Kommunikationsstruktur die von den {\"u}blichen Protokollen 
abweicht. Das Content Relevance-Ortiented Protocol (CROP) soll hier
f{\"u}r Abhilfe schaffen.
Der Gedanke des Protokolls ist ein gezielter Umgang mit den begrenzten 
Ressourcen. Hierbei wird einerseits eine Verallgemeinerung der vorhandenen 
Daten in Datenbl{\"o}cke vorgenommen und andererseits eine Priorisierung der 
Daten erwirkt, welche {\"u}ber die Reihenfolge und den Zeitpunkt des 
Versendens entscheidet. Daraus resultiert, dass Daten von hoher Relevanz 
vor Daten mit niederer Relevanz verschickt werden. Somit k{\"o}nnen 
beispielsweise zeitkritische Daten das Kontrollzentrum auf der Erde 
schneller erreichen.
\todo{kurze zusammenfassung zu den eizelnen kapiteln}
