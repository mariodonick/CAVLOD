Für die interplanetare Kommunikation stehen besondere Aspekte, wie Latenz,
Bandbreite oder der lokale Speicher im Vordergrund. Dazu ist ein besonderes
Kommunikationsprotokoll notwendig. Für dieses Protokoll existiert
bereits ein Konzept (\cite{Daher}), welches im Rahmen dieser Arbeit
teilweise verfeinert, implementiert und bewertet werden soll.

Dabei wird in Kapitel \ref{cap:standDerTechnik} und \ref{cap:grundlagen} ein
erster Überblick über den momentanen Stand der Technik sowie Grundlagen zur
Einordnung der Thematik gegeben.

Die konzeptionelle Darstellung des Protokolls erfolgt in Kapitel
\ref{cap:konzept}. Dazu gehören neben möglichen Implementierungen und
Vorüberlegungen auch die eigentliche theoretische Umsetzung sowie Szenarien,
welche den Ablauf des Protokolls beispielhaft vorstellen. Der eigentliche
softwaretechnische Aufbau und die Implementierung werden in Kapitel
\ref{cap:implementierung} detailliert beschrieben. Abschließend wird das
Protokoll hinsichtlich Overhead, Laufzeit und Speicherbedarf analysiert und
ausgewertet (Kapitel \ref{cap:protokollAnalyse}).

Das in dieser Arbeit umgesetzte Protokoll bietet eine solide
Grundlage zur relevanzorientierten Kommunikation. Jedoch besteht noch Raum
für Erweiterungen und Optimierungen.

Weiterhin ist eine Optimierung der implementierten Algorithmen sinnvoll.
Der erste Anhaltspunkt ist das Submodul \Code{StoreManager}.
Die Festplattenzugriffe während des kontinuierlichen Speicherns und Ladens der Daten
sind sehr langsam und bremsen das gesamte Modul. Eine Möglichkeit, dies zu
verbessern, wäre, das Modul in einem extra Prozess auszuf{\"u}hren. Somit sind
die Festplattenzugriffe unabhängig vom eigentlichen Programmablauf. Ein
weiterer Ansatzpunkt ist die Datenstruktur \Code{SmartPrioritizedQueue}. Für diese muss
eine effiziente Möglichkeit gefunden werden, Datenblöcke schnell
einzusortieren, zu löschen und Bl{\"o}cke bestimmter Größe zu finden.
Die angedachte Kompressionseinstellung aus Kapitel \ref{sec:ProtokolDesign}
findet beispielsweise im \Code{StoreManager} schon Berücksichtigung, wurde aber
noch nicht vollständig implementiert. Dadurch kann in Zukunft für viele kleine
Datenblöcke weiterer Overhead eingespart werden.

Das Protokoll ist aufgrund seiner Flexibilität auf eine Vielzahl an Datentypen
anwendbar. Die Handhabung vieler Daten eines Datentyps zur selben Zeit können
in Form von parallel ablaufenden Prozessen zur Geschwindigkeitsoptimierung
beitragen. Um andere Datentypen im Protokoll zu verwenden, sollte eine
Schnittstelle zum Laden von Plugins implementiert werden. Damit besteht dann die
Möglichkeit, dem Protokoll Algorithmen hinzuzufügen, wodurch
übliche Datenformate wie docx oder bmp verarbeitet werden können.

Die ebenfalls im Rahmen dieser Arbeit entwickelte ChatGui in Kapitel
\ref{cap:chatGui} soll einen ersten Einstieg in die benutzerfreundliche
Verwendung des Protokolls geben. In dieser kann das Zusammenführen von
Textfragmenten noch optimiert werden. Interessant ist auch eine Anzeige
über die ablaufende Zeitdauer, nach welcher ein versendetes Paket den
Empfänger erreicht. Um das entwickelte Modul in Serverumgebungen zu integrieren,
muss ein Programm entwickelt werden, welches in der Lage ist, dieses als
Webservice zu starten.
Dadurch können schon bestehende Benutzeroberflächen das Protokoll nutzen.
