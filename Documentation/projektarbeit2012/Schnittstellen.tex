Anhand der im vorherigen Kapitel \ref{sec:Anwendungsszenarien} genannten Anwendungsszenarien lassen
sich die einzelnen Schnittstellen zwischen den Modulen ableiten. In
\abbildung{Schnittstellen} ist ein schematischer Aufbau der Module
dargestellt. Die Schnittstellen zwischen diesen sind numerisch gekennzeichnet.

\begin{figure}[H]
\centering
\includegraphics[scale=.7]{Schnittstellen.pdf}
\caption{Schnittstellen der Module}
\label{fig:Schnittstellen}
\end{figure}

Daraus ergeben sich die folgenden Verallgemeinerungen:

\textbf{Schnittstelle (1)}\newline
Die Schnittstelle (1) ist die Gesamtschnittstelle. Hier wird der
Content an das Modul übergeben. Weiterhin erfolgt die Übergabe der \gls{DOID}, des
Datentyps und eines Flags, welches angibt, ob ein Zeitangabe berücksichtigt
werden soll.
Optional sind ein oder mehrere zeichengenaue Positionsangaben mit einem
Relevanzwert anzugeben. Der Relevanzwert weist hierbei eine
prozentuale Angabe im Bereich von 0 bis 100 auf. Dabei wird 0 als unwichtig und
100 als sehr wichtig eingestuft. Die Positionsangaben beinhalten zum einen den
Startpunkt des relevanten Bereiches mit einer x- und y-Koordinate. Weiterhin
wird die Fläche des Bereiches mit einer Länge in x- und y-Richtung angegeben.
Die Angabe der Positionen und des Relevanzwertes ist beispielsweise notwendig,
wenn diese Daten nicht selbst berechnet werden sollen, sondern der Benutzer
diese über eine grafische Benutzeroberfläche vorgibt.

\textbf{Schnittstelle (2)} 

An den Evaluator wird von den Eingangsdaten der Schnittstelle (1)
der Content übergeben. Zusätzlich werden die Relevanzwerte mit den
Positionsangaben übermittelt, wenn diese vorgegeben wurden.

\textbf{Schnittstelle (3)} 

Aus Schnittstelle (1) wird an die Partitonierung
die \gls{DOID}, der Datentyp, der Content und das Zeitflag weitergegeben.

\textbf{Schnittstelle (4)} 

Damit der Evaluator nachvollziehen kann, welches Datenpaket gerade bearbeitet
wird, erfolgt die Übergabe der \gls{DOID} und des Datentyps.  

\textbf{Schnittstelle (5)}

Als Antwort auf die eingehenden Daten der Schnittstelle (4)
erfolgt die Rückgabe der relevanten Bereiche mit den Positionen und den
dazugehörigen Relevanzwerten durch den Evaluator.

\textbf{Schnittstelle (6)} 

Nach der Zerlegung der Blöcke werden die einzelnen
Datenblöcke mit den dazugehörigen Datenblockheadern an die
Priorisierung übergeben.

\textbf{Schnittstelle (7)} 

Durch Übergabe der \gls{DOID}, des Datentypes und der Positionsangaben kann der
Content identifiziert werden.

\textbf{Schnittstelle (8)}

Durch erhalt der Daten in Schnittstelle (7) kann für
den Datenblock eine Priorität berechnet werden, welche an die Priorisierung übergeben
wird. Die Priorität wird als reelle Zahl aufgefasst und ist ähnlich des
Relevanzwertes eine prozentuale Größe zwischen 0 und 100.

\textbf{Schnittstelle (9)} 

Nach der Priorisierung werden die priorisierten Datenblöcke mit den
Datenblockheadern und der Priorität an das Verpackungsmodul übergeben.

Bei den Schnittstellen ist zu beachten, dass manche Angaben abhängig von den
jeweiligen Datentypen sind und für den Fall nicht benutzt oder
standardmäßig auf einen bestimmten Wert gesetzt werden. Beispielsweise benötigt
ein Text als Positionsangaben keine Länge in y-Richtung. Dieser ist bei einem
Bild jedoch zwingend notwendig.
