% Meta-Informationen -----------------------------------------------------------
%   Definition von globalen Parametern, die im gesamten Dokument verwendet
%   werden k�nnen (z.B auf dem Deckblatt etc.).
%
%   ACHTUNG: Wenn die Texte Umlaute oder ein Esszet enthalten, muss der folgende
%            Befehl bereits an dieser Stelle aktiviert werden:
%            \usepackage[latin1]{inputenc}
% ------------------------------------------------------------------------------
\newcommand{\titel}{Implementierung und Bewertung eines Konzeptes zur "Content
					  Relevance-Oriented Data Transport (CRODT)" \"uber Interplanetary
					  Communication Networks}
\newcommand{\untertitel}{und hier kommt der Untertitel}
\newcommand{\art}{Projektarbeit}
\newcommand{\fachgebiet}{Software-Engineering}
\newcommand{\autor}{Marian Haescher, Florian Ludwig, Philip Grunert}
\newcommand{\studienbereich}{}
\newcommand{\matrikelnrFLORIAN}{12 34 56}
\newcommand{\matrikelnrMARIAN}{12 34 56}
\newcommand{\matrikelnrPHILIP}{7200626}
\newcommand{\erstgutachter}{Prof. Dr.-Ing. habil. Djamshid Tavangarian}
\newcommand{\zweitgutachterA}{Dr.-Ing. Robil Daher}
\newcommand{\zweitgutachterB}{Mario Donick, M.A.}
\newcommand{\jahr}{2012}
\newcommand{\ort}{Rostock}
\newcommand{\logo}{UNI-Logo.pdf}
