\label{cap:konzept}

In diesem Kapitel werden vorrangig {\"U}berlegungen zur Fehlererkennung
und der Handhabung von Verbindungsabbr{\"u}chen bzw. fehlerhaften
{\"U}bertragungen dargelegt. In diesem Kontext werden die Notwendigkeiten einer
\gls{TTL} sowie die unterschiedlichen Optionen zur
Realisierung einer Datenkonsistenzpr{\"u}fung via CRC-Checksumme erl{\"a}utert.
Des Weiteren werden {\"U}berlegungen bez{\"u}glich der Entwicklung des
\gls{ROTP}-Stack aufgezeigt und analysiert. Dafür wird ein Protokoll
entwickelt, welches unterschiedliche Datenformate verpackt. Für die anstehende
Implementierung werden ausgehend von verschiedenen Anwendungsszenarien die
Schnittstellen der notwendigen Module bestimmt. Als Grundlage aller
{\"U}berlegungen diente das CRODT-Framework \cite{Daher}.
