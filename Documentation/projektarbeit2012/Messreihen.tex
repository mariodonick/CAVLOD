\label{subCap:Messreihen}

Die Messungen wurden mit einem Notebook aufgenommen, welches
folgende Hardwarekonfigurationen besitzt.

\textit{
	LENOVO Thinkpad t410 Modell 2522-w29 \newline
	\textbf{Prozessor}: Intel Core i5 520M (2 Kerne mit Hyperthreading $\stackrel{\wedge}=$ 4 Threads)\\
	\textbf{Arbeitsspeicher}: 4 GB DDR3	
	}

Da eine ähnliche Hardware bei einer Verwendung des Protokolls auf dem Mars
weniger zu erwarten ist, sind die Größenordnungen der nachfolgenden Ergebnisse
nur ein Richtwert. Für die folgenden Betrachtungen sind die Beziehungen und
Entwicklungen der einzelnen Werte zueinander wichtig, welche mit schwächerer
Hardware nahezu identisch sind.

Das Protokoll wird grundlegend mit zwei verschiedenen
Compilereinstellungen getestet. Dies sind die Geschwindigkeits- (\glqq O$2$
\grqq) und die Quellcodegrößenoptimierung (\glqq Os \grqq). Dabei
wurde untersucht, wie lange das Protokoll für die Verarbeitung von Texten
verschiedener Größe benötigt, die keine relevanten Bereiche besitzen. Dafür
wurden Dateien mit verschiedenen Größen von $2$ Bytes bis $10$ Millionen Bytes 
angelegt, durch das Programm eingelesen und der Schnittstelle des Protokolls
übergeben. In diesem Fall ist keine Vorpriorisierung notwendig und die Datei
kann sukzessive in maximale Datenblockgrößen zerlegt werden.
Im Vergleich dazu ist eine Messreihen mit unterschiedlich vielen relevanten
Bereichen für einen Text mit der Größe von $5.000.000$ Byte aufgenommen
worden.
Durch das Anfallen von zusätzlichen Berechnungen, die zum Zerteilen 
des Textes notwendig sind, ist zu erwarten, dass die Laufzeit gegenüber
ersterer Betrachtung ansteigt.

Die eben vorgestellten Messungen wurden auf zwei verschiedenen Wegen
aufgenommen. Diese unterscheiden sich dahingehend, dass das Modul vor dem Erhalt
neuer Daten gelöscht und anschließend neu erstellt wurde. Somit sind alle
Datenstrukturen bei jeden Messdurchgang leer, wodurch vor allem die Komplexität
der Einsortierung in die priorisierende \gls{FIFO} verringert wurde. Für den
zweiten Fall wurden die Daten alle nacheinander verschickt, um ein realistischeres
Szenario darstellen zu können.

Für einen aussagekräftigeren Wert und um potentielle Messungenauigkeiten zu
kompensieren, wurde die Messung für jede Einstellung $10$ Mal wiederholt und der
Mittelwert gebildet. Die Messreihen sind zur besseren Übersicht im
Anhang dargestellt (\ref{sec:messtabellen}).