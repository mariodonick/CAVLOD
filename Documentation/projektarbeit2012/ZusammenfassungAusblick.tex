% Zusammenfassung

Für die interplanetare Kommunikation stehen besondere Aspekte, wie Latenz,
Bandbreite oder der lokale Speicher im Vordergrund. Dazu ist ein besonderes
Kommunikationsprotokoll notwendig. Für dieses Protokoll existiert
bereits ein Konzept \cite{Daher}, welches im Rahmen dieser Arbeit
teilweise verfeinert, implementiert und bewertet wurde.

Das Kapitel \ref{cap:grundlagen} beschäftigte sich
mit den momentanen Stand der Technik sowie den Grundlagen zur Einordnung der
Thematik. Dabei wurde eine Übersicht über die grundlegenden Begriffe im
Bereich der Kommunikation gegeben. Weiterhin wurde die Bedeutung und
Auswirkung eines geringen Zeitslots für die Kommunikation zwischen Mars und Erde
verdeutlicht.\newline
Einen umfassenden Überblick über aktuelle Technologien, welche auf heutigen
Marsmissionen eingesetzt werden, werden in Kapitel \ref{cap:standDerTechnik}
dargelegt.
Dabei wird neben der verbauten Technik der Marsrover auch bisher verwendete
Kommunikationsprotokolle vorgestellt.

In Kapitel \ref{cap:konzept} wurden weitere Ansätze und Ideen für das Konzept
\gls{CRODT}/\gls{CROP} vorgestellt dieses zu verfeinern. Dabei wurde die
Möglichkeit der Verwendung einer \gls{TTL}, einer Datenkompression und einer
Prüfsumme diskutiert. Für die Realisierung wurde ein Protokoll-Design
entwickelt. Darin wurden die optimale Größen und Anordnungen der Daten erörtert,
um ein Ver- und Entpacken unterschiedlicher Datenformate auf beiden Seiten der
Kommunikation zu ermöglichen.
Schlussendlich wurde anhand von Anwendungsszenarien die Schnittstellen der
Module abgeleitet, welche in der Realisierung verwendet wurden.

Der softwaretechnische Aufbau und die Realisierung des Konzepte
wurde in Kapitel \ref{cap:implementierung} detailliert beschrieben. Dabei wurde
jeweils ein Modul zum Empfangen und Senden implementiert.
Dabei wurden auch Verbindungsabbrüche und Systemabstürze berücksichtigt, worauf
das Modul \Code{StoreManager} entwickelt wurde. Dieses ist in der Lage nach
einem Fehler den vorherigen Zustand wiederherzustellen.
Weiterhin wurde eine Kommunikationsumgebung in Form eines ChatGui entwickelt.
Unter Verwendung des \gls{ROTP} können hier Prioritäten vom Benutzer selbst
festgelegt und vom CROP verarbeitet werden. Damit steht eine Beispielanwendung
zur Verfügung, mit dessen Hilfe sich unter anderem die Effizienz von
M$2$M-Kommunikation des \gls{CROP}-Protokolls durchführen läßt.

Abschließend wurde im Kapitel \ref{cap:protokollAnalyse} das Protokoll
hinsichtlich Overhead, Laufzeit und Speicherbedarf analysiert und ausgewertet.
Dabei kamen zwei verschiedene Compilereinstellungen zum Einsatz, die entweder
die Größe des Quellcodes oder die Laufzeit des Programms beeinflussen. Die
Ergebnisse der Analyse zeigten wie wichtig eine sinnvolle Priorisierung durch
die CROP ist und wie stark der Overhead und die Laufzeit von dieser abhängen.

% Ausblick
Das in dieser Arbeit umgesetzte Protokoll bietet eine solide
Grundlage zur relevanzorientierten Übertragung. Jedoch besteht noch Raum
für Erweiterungen und Optimierungen.

So ist z.B. eine Optimierung der implementierten Algorithmen sinnvoll.
Der erste Anhaltspunkt ist das Submodul \Code{StoreManager}.
Die Zugriffe auf ein Speichermedium während des kontinuierlichen Speicherns und
Ladens der Daten sind sehr langsam und bremsen das gesamte Modul. Eine
Möglichkeit, dies zu verbessern, wäre, das Modul in einem extra Prozess
auszuf{\"u}hren. Dann wären die Zugriffe unabhängig vom eigentlichen
Programmablauf. Ein weiterer Ansatzpunkt ist die Datenstruktur
\Code{SmartPrioritizedQueue}. Für diese muss eine effiziente Möglichkeit
gefunden werden, Datenblöcke schnell einzusortieren, zu löschen und Bl{\"o}cke
bestimmter Größe zu finden.
Die angedachte Kompressionseinstellung aus Kapitel \ref{sec:ProtokolDesign}
findet beispielsweise im \Code{StoreManager} schon Berücksichtigung, wurde aber
noch nicht vollständig implementiert. Dadurch kann in Zukunft für viele kleine
Datenblöcke weiterer Overhead eingespart werden.

Das Protokoll ist aufgrund seiner Flexibilität auf eine Vielzahl an Datentypen
anwendbar. Die Handhabung vieler Daten eines Datentyps zur selben Zeit kann
in Form von parallel ablaufenden Prozessen zur Geschwindigkeitsoptimierung
beitragen. Um andere Datentypen im Protokoll zu verwenden, sollte eine
Schnittstelle zum Laden von Plugins implementiert werden. Damit besteht dann die
Möglichkeit, dem Protokoll Algorithmen hinzuzufügen, wodurch
übliche Datenformate wie docx oder bmp verarbeitet werden können.

Das ebenfalls im Rahmen dieser Arbeit entwickelte ChatGui in Kapitel
\ref{cap:chatGui} hinsichtlich des Zusammenführens der einzelnen Textfragmente
zu optimiert. Sinnvoll wäre hier eine Anzeige über die ablaufende Zeitdauer,
nach welcher ein versendetes Paket den Empfänger erreicht.

Um das entwickelte Modul in Serverumgebungen zu integrieren, muss ein Programm
entwickelt werden, welches in der Lage ist, dieses als Webservice zu starten.
Dadurch können schon bestehende Benutzeroberflächen das Protokoll nutzen.
