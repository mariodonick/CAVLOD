\label{cap:konzept}

Dieses Kapitel befasst sich mit den essenziellen Vorbetrachtungen zur
Projektarbeit. Dabei geht es vorrangig um {\"U}berlegungen zur Fehlererkennung
und der Handhabung von Verbindungsabbr{\"u}chen bzw. fehlerhaften
{\"U}bertragungen. In diesem Kontext werden die Notwendigkeit einer TTL sowie
die unterschiedlichen Optionen zur Realisierung einer Datenkonsistenzpr{u}fung
via CRC-Checksumme er{\"o}rtert. Desweiteren werden {\"U}berlegungen
bez{\"u}glich der Entwicklung des CROP Protokollstacks aufgezeigt und
analysiert. Dafür wird ein Protokol designt, welches unterschiedliche
Datenformate \todo{besser formulieren/ergänzen PHIL!?} verpackt. Für die
anstehende Implementierung werden ausgehend von Anwendungszenarien die
Schnittstellen der notwendingen Module bestimmt.