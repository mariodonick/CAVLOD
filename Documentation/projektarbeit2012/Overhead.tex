Wie bereits im Kapitel \ref{sec:ProtokolDesign} erläutert, existiert neben den
eigentlichen Daten der Header. Die zusätzliche Größe, um die sich die
zu versendende Nachricht dadurch erhöht, heißt Overhead. Die Berechnung und
Bedeutung dessen wird im Folgenden beschrieben.

Ausgehend vom Aufbau einer Nachricht (siehe Abbildung
\ref{fig:uebersichtdatenaufschluesselung}) sind bei der Berechnung des Overhead
zwei Header zu betrachten. Einerseits ist dies der allgemeine Header der
Nachricht und zum Anderen trägt jeder einzelne Datenblock
mit einem eigenen Header zum Overhead bei. Dies kann durch die Formel
\ref{eq:overhead} ausgedrückt werden.

\begin{equation}
	Overhead = Overhead_{Nachricht} + n * Overhead_{Datenblock}
	\label{eq:overhead}
\end{equation}

Dabei variiert die Größe des Nachrichtenheaders je nach gewähltem
Übertragungsprotokoll (Formel \ref{eq:overheadMessage}). In dem Protokoll,
welches in dieser Arbeit vorgestellt wird, ist dies das IPv6. Das IPv6
reserviert $32$ Byte im Header. Damit ergibt sich für diesen speziellen Fall $37$ Byte für
den Nachrichtenheader. Ein Datenblockheader besteht immer konstant aus $9$ Byte.
Die angedeutete Kompressionseinstellung findet in dieser Arbeit noch
keine Berücksichtigung.

\begin{equation}
	Overhead_{Nachricht} = 5 Byte + Address_{Source} + Address_{Destination}
	\label{eq:overheadMessage}
\end{equation}

Aus der Formel \ref{eq:overhead} ist zu erkennen, dass der Overhead
stetig mit der Anzahl an gleichzeitig zu verschickenden Datenblöcken ($n$)
steigt. Einzig der Nachrichtenoverhead hat für große $n$ nur noch eine
geringe Auswirkung. Somit ist ein Versenden einer Nachricht mit vielen Daten
gleichzeitig sinnvoller, um überflüssigen Overhead zu vermeiden. Dieser
Effekt wird mit Formel \ref{eq:overheadRatio} noch einmal verdeutlicht. Diese
gibt das Verhältnis der Größe von Overhead zur gesamten Nachricht an und
ermöglicht eine prozentuale Aussage.

\begin{eqnarray} 
	overhead_{ratio} & = & \frac{Overhead}{Nachricht_{gesamt}}\\
	overhead_{ratio} & = & \frac{Overhead_{Nachricht} + n * Overhead_{Datenblock}}{Overhead_{Nachricht} + n * (Overhead_{Datenblock} + Content)}\\
	overhead_{ratio} & = & \frac{Overhead}{Overhead + n * Content}
	\label{eq:overheadRatio}
\end{eqnarray}