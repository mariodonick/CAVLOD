Wie bereits im Kapitel \ref{sec:ProtokolDesign} erläutert, existiert neben den
eigentlichen Daten der Header. Die zusätzliche Größe, um die sich die
zu versendende Nachricht dadurch erhöht, heißt Overhead. Die Berechnung und
Bedeutung dessen wird im Folgenden beschrieben.

Ausgehend vom Aufbau einer Nachricht (siehe Abbildung
\ref{fig:uebersichtdatenaufschluesselung_text}) sind bei der Berechnung des
Overhead zwei Header zu betrachten. Einerseits ist dies der allgemeine Header der
Nachricht und zum anderen trägt jeder einzelne Datenblock
mit einem eigenen Header zum Overhead bei. Dies kann durch die Formel
\ref{eq:overhead} ausgedrückt werden.

\begin{equation}
	Overhead = Overhead_{Nachricht} + n * Overhead_{Datenblock}
	\label{eq:overhead}
\end{equation}

Dabei variiert die Größe des Nachrichtenheaders je nach gewähltem
Übertragungsprotokoll (Formel \ref{eq:overheadMessage}). In dem Protokoll,
welches in dieser Arbeit vorgestellt wird, ist dies das IPv6. Das IPv6
reserviert $32$ Byte im Header. Damit ergeben sich für diesen speziellen
Fall $37$ Byte für den Nachrichtenheader. Ein Datenblockheader besteht immer
konstant aus $9$ Byte.
Die angedeutete Kompressionseinstellung findet in dieser Arbeit noch
keine Berücksichtigung.

\begin{equation}
	Overhead_{Nachricht} = 5 Byte + Address_{Source} + Address_{Destination}
	\label{eq:overheadMessage}
\end{equation}

Aus der Formel \ref{eq:overhead} ist zu erkennen, dass der Overhead
stetig mit der Anzahl an gleichzeitig zu verschickenden Datenblöcken ($n$)
steigt. Einzig der Nachrichtenoverhead hat für große $n$ nur noch eine
geringe Auswirkung. Somit ist ein Versenden einer Nachricht mit vielen Daten
gleichzeitig sinnvoller, um überflüssigen Overhead zu vermeiden. Dieser
Effekt wird mit Formel \ref{eq:overheadRatio} noch einmal verdeutlicht. Diese
gibt das Verhältnis der Größe von Overhead zur gesamten Nachricht an und
ermöglicht eine prozentuale Aussage.

\begin{eqnarray} 
	overhead_{ratio} & = & \frac{Overhead}{Nachricht_{gesamt}}\\
	overhead_{ratio} & = & \frac{Overhead_{Nachricht} + n * Overhead_{Datenblock}}{Overhead_{Nachricht} + n * (Overhead_{Datenblock} + Content)}\\
	overhead_{ratio} & = & \frac{Overhead}{Overhead + n * Content}
	\label{eq:overheadRatio}
\end{eqnarray}

Der vorgestellte Sachverhalt soll im Folgenden an einem Beispiel
verdeutlicht werden. Hierzu wird wie ein Text mit $1.000.000$ Byte herangezogen.
Zunächst soll der Text ohne relevante Bereiche übertragen werden. Das heißt,
dass für den kompletten Inhalt nur ein einziger Datenblock benötig wird. Damit
ergibt sich, ausgehend von Formel \ref{eq:overheadRatio} folgendes Bild für den
Overhead ($Overhead_{worp}$).

% \begin{eqnarray} 
% 	Overhead_{worp} & = & Overhead_{Nachricht} + n * Overhead_{Datenblock} \\
% 	Overhead_{worp} & = & Overhead_{Nachricht} + Overhead_{Datenblock} \\
% 	Overhead_{worp} & = & 37 Byte + 9 Byte \\
% 	Overhead_{worp} & = & 46 Byte
% 	\label{eq:overheadRatio_worp}
% \end{eqnarray}

\begin{eqnarray} 
	Overhead_{worp} & = & \frac{Overhead}{Nachricht_{gesamt}}\\
	Overhead_{worp} & = & \frac{Overhead_{Nachricht} + n * Overhead_{Datenblock}}{Overhead_{Nachricht} + n * (Overhead_{Datenblock} + Content)}\\
	Overhead_{worp} & = & \frac{46 Byte}{46 Byte + 1.000.000 Byte} \\
	Overhead_{worp} & = & 0,46 \cdot 10^{-2} \%
	\label{eq:overheadRatio_worp}
\end{eqnarray}

Der selbe Text wird jetzt mit $50$ wichtigen Fragmenten versehen. Damit ändert
sich der Overhead ($Overhead_{wrp}$) wie in Formel \ref{eq:overheadRatio_wrp}
dargestellt ab.

% \begin{eqnarray} 
% 	Overhead_{wrp} & = & Overhead_{Nachricht} + n * Overhead_{Datenblock} \\
% 	Overhead_{wrp} & = & Overhead_{Nachricht} + 50 * Overhead_{Datenblock} \\
% 	Overhead_{wrp} & = & 37 Byte + 50 * 9 Byte \\
% 	Overhead_{wrp} & = & 487 Byte
% 	\label{eq:overheadRatio_wrp}
% \end{eqnarray}

\begin{eqnarray} 
	Overhead_{wrp} & = & \frac{Overhead}{Nachricht_{gesamt}}\\
	Overhead_{wrp} & = & \frac{Overhead_{Nachricht} + n * Overhead_{Datenblock}}{Overhead_{Nachricht} + n * (Overhead_{Datenblock} + Content)}\\
	Overhead_{wrp} & = & \frac{487 Byte}{487 Byte + 1.000.000 Byte} \\
	Overhead_{wrp} & = & 4,8 \cdot 10^{-2} \%
	\label{eq:overheadRatio_wrp}
\end{eqnarray}

Eine Gegenüberstellung der beiden Werte (Formel \ref{eq:overheadRatio_wrp_worp})
zeigt, dass sich das Einfügen von relevanten Bereichen nachteilig auf den
Overhead auswirkt. Je mehr dieser Fragmente vorhanden sind, desto größer wird
die zu übertragende Nachricht.

\begin{eqnarray} 
	\frac{ Overhead_{wrp} }{ Overhead_{worp} } & = & \frac{4,8 \cdot 10^{-2} \%}{0,46 \cdot 10^{-2} \%} \\
	\frac{ Overhead_{wrp} }{ Overhead_{worp} } & = & 10,5 \\ 
	Overhead_{wrp} & = & 10,5 \cdot Overhead_{worp}
	\label{eq:overheadRatio_wrp_worp}
\end{eqnarray}

Somit ist beim Versand eines Texten mit $50$ relevanten Bereichen mit einem fast
elf mal so hohem Overhead zu rechnen. Trotz der großen Erhöhung ist der Overhead
sehr niedrig und kommt gegenüber der zu verschickenden Datenmenge kaum zur
Geltung. Dies zeigt, dass das entwickelte Protokoll sehr effizient arbeitet.
