Für die interplanetare Kommunikation stehen besondere Aspekte, wie Latenz,
Bandbreite oder der lokale Speicher, im Vordergrund. Dazu ist ein besonderes
Kommunikationsprotokoll notwendig. Für dieses Protokoll \todo{LINK} existierte
bereits ein Konzept, welches im Rahmen dieser Arbeit verfeinert, implementiert
und bewertet werden soll.

Dabei wird in Kapitel \ref{cap:standDerTechnik} und \ref{cap:grundlagen} ein
erster Überblick über den momentanen Stand der Technik, sowie Grundlagen zur
Einordnung der Thematik gegeben \todo{kommentar}
% ob das nu so wirklich stimmt
% weiß ich net ... klingt erstmal toll aber vllt muss da noch was anderes gefunden werden}
.
Die konzeptionelle Darstellung des Protokolls erfolgt in Kapitel
\ref{cap:konzept}. Dazu gehören neben möglichen Implementierungen und
Vorüberlegungen auch die eigentliche theoretische Umsetzung sowie Szenarien,
welche den Ablauf des Protokolls beispielhaft vorstellen. Der eigentliche
softwaretechnische Aufbau und die Implementierung wurde in Kapitel
\ref{cap:implementierung} detailliert beschrieben. Abschließend wurde das
Protokoll hinsichtlich Laufzeit und Speicherbedarf analysiert und ausgewertet
(Kapitel \ref{cap:auswertung}). \todo{kommentar}
% hier
% müssen noch 1-2 sätze hin was bei der auswertung rauskam, kommt quasi am ende ... kapitel
% ist am fertig werden ;)}

Das in dieser Arbeit umgesetzte Protokoll bietet eine solide
Grundlage zur Relevanz-Orientierten Kommunikation. Jedoch besteht noch Raum
für Erweiterungen und Optimierungen.

Zum Einen wäre eine Optimierung der implementierten Algorithmen sinnvoll.
Der erste Anhaltspunkt ist das Submodul \Code{StoreManager}.
Die Festplattenzugriffe während des kontinuierlichen Speicherns und Ladens der Daten
ist sehr langsam und bremst das gesamte Modul. Eine Möglichkeit dies zu
verbessern wäre das Modul in einem extra Prozess laufen zu lassen. Somit wären
die Festplattenzugriffe unabhängig des eigentlichen Programmes. Ein weiterer
Ansatzpunkt ist die Datenstruktur \Code{SmartPrioritizedQueue}. Für diese muss
eine effiziente Möglichkeit gefunden werden Datenblöcke schnell
einzusortieren, zu löschen und welche mit bestimmter Größe zu finden.

Das Protokoll ist auf Grund seiner Flexibilität auf eine Vielzahl an Datentypen
anwendbar. Die Handhabung vieler Daten eines Datentyps zur selben Zeit könnten
in Form von parallel Ablaufenden Prozessen zur Geschwindigkeitsoptimierung
beitragen. Um andere Datentypen vom Protokoll zu verwenden sollte eine
Schnittstelle zum Laden von Plugins implementiert werden. Damit besteht dann die
Möglichkeit dem Protokoll Algorithmen hinzuzufügen, wodurch
übliche Datenformate wie docx oder bmp verarbeitet werden können.

Die ebenfalls in diesem Rahmen der Arbeit entwickelt ChatGui in Kapitel
\ref{cap:chatGui} soll einen ersten Einstieg in die benutzerfreundliche
Verwendung des Protokolls geben. In dieser kann das Zusammenführen von
Textfragmenten noch optimiert werden. Interessant wäre auch eine Anzeige
über die ablaufende Zeitdauer, wann ein versendetes Paket den Empfänger
erreicht. Um das entwickelte Modul in Serverumgebungen zu integrieren, muss ein
Programm entwickelt werden dieses als Webservice zu starten. Dadurch können
schon bestehende Benutzeroberflächen das Protokoll nutzen.
 
% !!!! WICHTIG Kommt auf dieser Seite nach ganz unten  für die QUELLEN ----> WICHTIG!!!!


%\bibliographystyle{plaindin}
%\bibliography{Referenzen}


