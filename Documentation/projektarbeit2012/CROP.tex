Das \gls{CROP} bezeichnet einen Protokoll-Stack, der zur relevanzabh{\"a}ngigen
Daten{\"u}bertragung in \gls{DTN} eingesetzt wird. Dabei
umfasst das daf{\"u}r entwickelte Framework sowohl das Splitten von beliebigen
Daten, als auch eine Evaluierung selbiger. Die Evaluierung geschieht entweder
automatisch (z.B. mit einer Ontologie), oder manuell durch die menschlichen
Kommunikationspartner, die unter Nutzung von CROP miteinander kommunizieren
\cite{Daher}.

\section{Sende-/Empfangszeitfenster sowie Speicherung der Daten am Beispiel
Mars}

Wie bereits in der Einleitung erw{\"a}hnt, bleiben einem Rover wie z.B.
\textit{Curiosity} auf dem Mars nach aktuellen Angaben der NASA \cite{web5}
lediglich zwei Zeitfenster mit je 15 Minuten zur {\"U}bertragung der gesammelten
Daten. Die dabei {\"u}bertragenen Daten werden vom \textit{NASA Mars
Reconnaissance Orbiter} empfangen und an das irdische \textit{Deep Space
Network} gesendet. Das \textit{Odyssey spacecraft} der NASA, welches
sich ebenfalls im Mars-Orbit befindet, dient zus{\"a}tzlich als
\textit{Backup\footnote{Backupdevice: Ger{\"a}t zur Zwischenspeicherung von
Daten}- bzw.
Relaydevice\footnote{Relaydevice: Ger{\"a}t zur Vermittlung von Daten zwischen
Rover und Erde}}. Die f{\"u}r eine Mars-Erde-Kommunikation notwendige
Zeitkonvertierung (z.B. bei der Ermittlung der Bedeutung eines Zeitstempels in
einem Datenpaket) wird im nachfolgenden Kapitel behandelt.
\label{Empfangsfenster}