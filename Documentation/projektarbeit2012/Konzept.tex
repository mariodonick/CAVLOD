\label{cap:konzept}

Dieses Kapitel befasst sich mit den essenziellen Vorbetrachtungen zur
Projektarbeit. Dabei werden vorrangig {\"U}berlegungen zur Fehlererkennung
und der Handhabung von Verbindungsabbr{\"u}chen bzw. fehlerhaften
{\"U}bertragungen dargelegt. In diesem Kontext werden die Notwendigkeit einer
TTL sowie die unterschiedlichen Optionen zur Realisierung einer Datenkonsistenzpr{u}fung
via CRC-Checksumme er{\"o}rtert. Desweiteren werden {\"U}berlegungen
bez{\"u}glich der Entwicklung des CROP Protokollstacks aufgezeigt und
analysiert. Dafür wird ein Protokoll designt, welches unterschiedliche
Datenformate \todo{besser formulieren/ergänzen PHIL!?} verpackt. Für die
anstehende Implementierung werden ausgehend von verschiedenen Anwendungszenarien
die Schnittstellen der notwendigen Module bestimmt.