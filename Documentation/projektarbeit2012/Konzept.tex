\label{cap:konzept}

In diesen Kapitel werden vorrangig {\"U}berlegungen zur Fehlererkennung
und der Handhabung von Verbindungsabbr{\"u}chen bzw. fehlerhaften
{\"U}bertragungen dargelegt. In diesem Kontext werden die Notwendigkeiten einer
\gls{TTL} sowie die unterschiedlichen Optionen zur
Realisierung einer Datenkonsistenzpr{\"u}fung via CRC-Checksumme erl{\"a}utert. Desweiteren
werden {\"U}berlegungen bez{\"u}glich der Entwicklung des \gls{CROP} Stack
aufgezeigt und analysiert. Dafür wird ein Protokoll entwickelt, welches
unterschiedliche Datenformate verpackt. Für die
anstehende Implementierung werden ausgehend von verschiedenen Anwendungszenarien
die Schnittstellen der notwendigen Module bestimmt. Als Grundlage aller
{\"U}berlegungen diente das CRODT-Framework (Ref. \cite{Daher}).
