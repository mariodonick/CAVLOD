% Zeilenabstand 1,5 Zeilen -----------------------------------------------------
\onehalfspacing

% Seitenraender
% -----------------------------------------------------------------
\setlength{\topskip}{\ht\strutbox} % behebt Warnung von geometry
%\geometry{paper=a4paper,left=35mm,right=35mm,top=30mm} % Ohne Binderand

\geometry{a4paper,inner=3.5cm,outer=2.5cm,top=2.5cm,bottom=2.5cm,pdftex} % Mit Binderand


% Kopf- und Fußzeilen ----------------------------------------------------------
\pagestyle{fancy} %scrheadings
% Kopf- und Fußzeile auch auf Kapitelanfangsseiten

\renewcommand*{\chapterpagestyle}{fancy}

%\renewcommand{\chapterpagestyle}{fancy} 
% Schriftform der Kopfzeile
%\renewcommand{\headfont}{\normalfont}

\fancyhead[ER]{\nouppercase{\sc\leftmark}}
\fancyhead[EL]{\nouppercase{\sc\rightmark}}
\fancyhead[OR]{\nouppercase{\sc\rightmark}}
\fancyhead[OL]{\nouppercase{\sc\leftmark}}

\fancyfoot[OL]{\copyright\ \autor}
\fancyfoot[OR]{\thepage}
\fancyfoot[EL]{\thepage}
\fancyfoot[ER]{\copyright\ \autor}
\fancyfoot[C]{}


%\renewcommand{\sectionmark}[1]{\markboth{#1}{}}

% sonstige typographische Einstellungen ----------------------------------------

% erzeugt ein wenig mehr Platz hinter einem Punkt
\frenchspacing 

% Schusterjungen und Hurenkinder vermeiden
\clubpenalty = 10000
\widowpenalty = 10000 
\displaywidowpenalty = 10000

% Quellcode-Ausgabe formatieren
\lstset{numbers=left, numberstyle=\tiny, numbersep=5pt, breaklines=true}
\lstset{emph={square}, emphstyle=\color{red}, emph={[2]root,base}, emphstyle={[2]\color{blue}}}

% Fußnoten fortlaufend durchnummerieren
\counterwithout{footnote}{chapter}
