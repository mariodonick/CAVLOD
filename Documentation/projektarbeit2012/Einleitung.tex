Seit {\"u}ber f{\"u}nfzig Jahren entwickeln Forscher Strategien zur Erkundung
des Weltraums. Dabei hat sich seit der Entsendung des ersten k{\"u}nstlichen
Satelliten namens Sputnik und der ersten Mondlandung viel getan.
Zahlreiche Sonden oder Landefahrzeuge wurden seither zu diversen Zielen unseres
Sonnensystems entsandt um neue Erkenntnisse {\"u}ber unser Universum und dessen
Entstehung zu sammeln. Ein aktuelles Beispiel stellt der Mars Rover Curiosity
dar, welcher j{\"u}ngst im Gale Krater auf dem Mars landete. Der Rover
verf{\"u}gt {\"u}ber zahlreiche Messinstrumente, darunter Kameras, Sensoren zur
Messung Kosmischer Strahlung, Spektrografen, Meteorologische Messinstrumente und
vieles mehr. Der Rover selbst ist mit einem Gewicht von $900$ kg und der
Gr{\"o}{\ss}e eines kompakten Kleinwagens das bisher mit Abstand gr{\"o}{\ss}te
vom Menschen geschaffene Objekt das je die Marsoberfl{\"a}che erreichte. Er
stellt ein mobiles Labor dar, das neben den Sensoren und Messinstrumenten
Greifarme und R{\"a}der zur Fortbewegung und Interaktion aufweist. Alle
Messinstrumente und s{\"a}mtliche Peripherie des Rovers werden dabei von der
Erde aus gesteuert. Mit steigender Komplexit{\"a}t der mobilen Labore
steigt nat{\"u}rlich auch die Menge an Daten die es zu {\"u}bertragen gilt. Da
die Komplette Steuerung und Auswertung auf der Erde stattfindet gilt es sowohl
Steuerdaten von der Erde an den Rover zu senden als auch Messdaten, Bilder und
{\"a}hnliches vom Rover an die Erde zu {\"u}bermitteln. Dabei unterliegt die
Kommunikation einigen entscheidenden Restriktionen. Die wohl wichtigste ist die
Lichtgeschwindigkeit, welche sich im Vakuum mit $299 792 458$ m/s ausbreitet.
Die Ausbreitung der entsandten Signale kann daher niemals schneller erfolgen. Da die
Entfernung zum Mars je nach Konstellation variiert, ist es im folgenden
hilfreich von der mittleren minimalen Entfernung zur Erde auszugehen. Diese
betr{\"a}gt $0,52$ AE oder aber ca. $77,8$ Millionen Kilometer (Schwankung des
$[Erde - Mars]$ Abstandes je nach Konstellation: $0,372 - 2,683$ AE, d.h. bei
Opposition $[Sonne - Erde - Mars]$ zwischen ca. $55 - 105$ Millionen Kilometern
und bei Konjunktion $[Mars - Sonne - Erde]$ ca. $400$ Millionen Kilometer).
Ausgehend von dieser Entfernung, bei einer Ausbreitungsgeschwindigkeit der
Signale von ca. $300 000$ km/s, erh{\"a}lt man eine Zeit von ca. $4,3$ Minuten
pro {\"U}bertragungsrichtung. Daraus folgt, dass die Antwort auf eine an den
Mars gesandte Frage (im aktuellen Szenario) im Idealfall nach $8,6$ Minuten die
Erde erreicht (im Fall der Maximalen Entfernung zwischen Erde und Mars sogar
erst nach $44$ Minuten, wobei in dieser Konstellation eine direkte
Funkverbindung nahezu unm{\"o}glich ist).
Da der Mars jedoch genau wie die Erde eine Rotation um seine eigene Achse ausf{\"u}hrt und die Kommunikation {\"u}ber eine
Marsstation{\"a}re Sendevorrichtung (Satellit) erfolgt, wird das Zeitfenster der
{\"U}bertragung weiter beschr{\"a}nkt. Diese Physikalisch begr{\"u}ndeten
Latenzen sind ein wichtiger Faktor den es in der Interplanetaren Kommunikation
zu ber{\"u}cksichtigen gilt. Die hieraus abzuleitende Erkenntnis ist, das eine
Kommunikation Beispielsweise zwischen Erde und Mars nicht mit einer heutigen
Irdischen Kommunikation zu vergleichen ist. Zwar ist das Modell der
verz{\"o}gerten Kommunikation dem Menschen keineswegs unbekannt, exemplarisch
sei hier auf den vor der Erfindung der Telegraphie ausschlie{\ss}lich
betriebenen Briefverkehr verwiesen, jedoch ist dieses Ph{\"a}nomen in der
heutigen vernetzten Generation nahezu in Vergessenheit geraten. Irdische
Kommunikation im $21$. Jahrhundert setzt auf geringe Latenzen und kann somit
idealisiert als verz{\"o}gerungsfrei betrachtet werden. Doch eine Interplanetare
Kommunikation wird nicht allein durch die unvermeidlichen Latenzen beeinflusst
auch die Bandbreite unserer {\"U}bertragung unterliegt zumindest auf der von der
Erde entfernten Seite einer Beschr{\"a}nkung. Am Beispiel der Mars Rover sei
hier auf die genutzten Solarpanele bzw. eingesetzten Batterien (zumeist
Radionuklidbatterien) verwiesen, welche nur eine begrenzte Energieversorgung
gew{\"a}hrleisten und somit keine nahezu unbegrenzte Sendeleistung 
erm{\"o}glichen wie es auf der Erde der Fall ist (vereinfacht betrachtet). 
Ein weiterer Punkt ist die f{\"u}r den Rover eingesetzte Technologie. Hierbei 
sind unter anderem Faktoren wie Gr{\"o}{\ss}e und Gewicht von Bedeutung, da 
diese die Kosten der Unternehmung stark beeinflussen. So kostet Beispielsweise 
der Start einer Ariane $5$ Rakete zwischen $125$ und $155$ Millionen Euro bei 
einer Nutzlast von $23000$ kg im LEO (Low Earth Orbit: $200-2000$ km). Hieraus 
ergibt sich f{\"u}r dieses Beispiel ein Preis von ca. $7000$ Euro pro kg 
Nutzlast. An diesem Beispiel wird sichtbar wie Wichtig es ist das Gewicht 
der Rover genau zu kalkulieren. Diese Restriktion wirkt sich direkt auf die 
verbaute Hardware des Rovers aus und somit auch indirekt auf Ressourcen wie 
z.B. die Gr{\"o}{\ss}e des verbauten Speichers um lokal Daten zu Speichern. 
Zusammenfassend kann man anhand der Beispiele feststellen, dass Latenz, 
Bandbreite sowie lokaler Speicher wichtige Restriktionen sind denen die 
Interplanetare Kommunikation unterliegt. Diese Vorraussetzungen verlangen 
eine besondere Kommunikationsstruktur die von den {\"u}blichen Protokollen 
abweicht. Das CROP Protokoll soll genau hier f{\"u}r Abhilfe sorgen. 
Der Gedanke des Protokolls ist ein gezielter Umgang mit den begrenzten 
Ressourcen. hierbei wird einerseits eine Verallgemeinerung der vorhandenen 
Daten in DataBlocks vorgenommen und andererseits eine Priorisierung der 
Daten erwirkt, welche {\"u}ber die Reihenfolge und den Zeitpunkt des 
Versendens entscheidet. Das Resultat ist nun, dass Daten Hoher Relevanz 
vor Daten einer niederen Relevanz verschickt werden. Somit k{\"o}nnen 
Beispielsweise zeitkritische Daten das Kontrollzentrum auf der Erde 
schneller erreichen. Als einfaches Beispiel dient an dieser Stelle ein 
Bild, welches einen Besonders relevanten Inhalt beherbergt, welcher 
zuerst versendet wird. Die Bildinhalte geringerer Relevanz werden erst 
dann gesendet, wenn keine weiteren Datenbl{\"o}cke h{\"o}herer Relevanz 
eines anderen Prozesses (Beispielsweise Sensorwerte) mehr vorliegen. Die 
Relevanz Verwaltung erfolgt somit Global {\"u}ber alle Daten (DataBlocks) 
und wird durch eine Ontologie gesteuert.
