Dieses Kapitel soll einen Überlick über das eigentliche Design des Protokolls
geben und indes auf den Aufbau und die Strukturierung der zu sendenden Nachricht
eingehen.

Hauptaugenmerk bei der Entwicklung eines Content Relevant-Oriented Protocol
(CROP) lag unter anderem auf dem sogenannten Header. Es galt einen zu großen
Overhead zu vermeiden. Der Header ist Bestandteil der Nachricht, in welcher
Informationen enthalten sind, die zum eindeutigen Versenden und Identifizieren
der mitgelieferten Daten notwendig sind. Dazu gehören zum Einen die
Versionanummer, die Configuration und die Länge der Nachricht. Zum Anderen die
genauen Adressen des Senders und Empfängers. Neben dem Header sind die
verpackten Daten, der sogenannte Payload, und ein CRC-Code zur Verschlüsselung
der Nachricht vorhanden. Im weiteren soll näher auf die eben genannten
Bestandteile eingegangen werden.

Die Versionsnummer nimmt die ersten vier Bit des Headers in Anspruch. Diese
signalisiert dem Empfänger mit welcher Version des Protokolls die Nachricht
versand wurde. Die nächsten $12$ Bit wurden für die Configuration reserviert.
Mit Hilfe dieser, können Einstellungen vorgenommen werden, die 