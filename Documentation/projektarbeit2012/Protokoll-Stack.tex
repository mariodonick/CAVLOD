Der \textit{Protokoll-Stack} (bzw. Protokollstapel) bezeichnet die
Verschachtelung der einzelnen Schichten, die zur Erf{\"u}llung einer
Protokollkommunikation notwendig sind. Dabei werden die Daten der vorherigen
Schicht zur Weiterverarbeitung der nachfolgenden Schicht zugef{\"u}hrt. Dieses
Stapelprinzip gibt dem \textit{protocol stack} seinen Namen. Der Stack
{\"u}bernimmt dabei s{\"a}mtliche Koordination und Verarbeitung der
Datenstr{\"o}me (z.B. Segmentierungen, Adressierung etc.).
Je nach Anforderung an die Kommunikation muss der Stack angepasst werden. Die
eingehenden Daten werden im Top-down Verfahren durch den Stack verpackt
und in einer \gls{FIFO} gepuffert. Nach dem Versenden werden die
Daten auf der Empfangsseite durch das Bottom-up Verfahren wieder ausgepackt und k{\"o}nnen
ausgewertet werden.
