Für die interplanetare Kommunikation stehen besondere Aspekte, wie Latenz,
Bandbreite oder der lokale Speicher, im Vordergrund. Dazu ist ein besonderes
Kommunikationsprotokoll notwendig. Ein derartiges Protokoll (CROP) war im Rahmen
dieser Arbeit zu entwickeln und vorzustellen.

Dabei wurde in Kapitel \ref{cap:standDerTechnik} und \ref{cap:grundlagen} ein
erster Überblick über den momentanen stand der Technik, sowie Grundlagen zur
Einordnung der Thematik gegeben \todo{ob das nu so wirklich stimmt weiß ich net
... klingt erstmal toll aber vllt muss da noch was anderes gefunden werden}.
Die konzeptionelle Darstellung des Protokolls erfolgte in Kapitel
\ref{cap:konzept}. Dazu gehörten neben möglichen Implementierungen und
Vorüberlegungen auch die eigentliche theoretische Umsetzung sowie Szenarien,
welche den Ablauf des Protokolls beispielhaft vorstellen. Der eigentliche
softwaretechnische Aufbau und die Implementierung wurde in Kapitel
\ref{cap:implementierung} detailliert beschrieben. Abschließend wurde das
Protokoll hinsichtlich Laufzeit und Speicherbedarf analysiert und ausgewertet
(Kapitel \ref{cap:testSzenario}, \ref{cap:auswertung}). \todo{hier müssen noch
1-2 sätze hin was bei der auswertung rauskam, kommt quasi am ende ... kapitel
ist am fertig werden ;)}

Das bis hierhin entwickelte Protokoll bietet eine solide Grundlage zur
Relevanz-Orientierten Kommunikation. Im Rahmen der weiteren Entwicklung und
Nutzung sind jedoch Verbesserungen notwendig. Hier kann an verschiedenen Stellen
angesetzt werden.

Zum Einen wäre eine Optimierung der implementierten Algorithmen sinmvoll.
Hauptaugenmerk sollte dabei unter anderem auf dem Storemanager liegen. Die
Festplattenzugriffe während des kontinuierlichen Speicherns und Ladens der Daten
geht sehr langsam von statten und bremst den restlichen Vorgang stark ab.

Das Protokoll ist auf Grund seiner Flexibilität auf eine Vielzahl an Datentypen
anwendbar. Die Handhabung vieler Daten eines Datentyps zur selben Zeit könnten
in Form von parallel Ablaufenden Prozessen zur Geschwindigkeitsoptimierung
beitragen.

Die ebenfalls in diesem Rahmen der Arbeit entwickelt ChatGui (Kapitel
\ref{cap:chatGui} soll nur einen ersten Einstieg in die benutzerfreundlichere
Verwendung des Protokolls geben. Dabei sind Verbesserungen in Form einer
überarbeiteten Usability notwendig.

Auf Grund der Tatsache, dass sich das Protokoll zur Zeit nur innerhalb eines
lokalen Netzes verwenden lässt, muss ein Weg gefunden werden, die Application
als Webservice starten zu lassen, sodass eine leichtes Verbinden auf Sender- und
Empfängerseite möglich ist.


ideen für ausblick:\newline
datentypen in form vieler dateien unterstützen (jpg doc etc) \newline
algorithmen zum umgang der dateien als plugins einbinden können \newline
kram als webservice starten und mit vorhandenen chat verknüppeln \newline
optimieren der algorithmen speziell des storemanagers (langsame
festplattenzugriffe) \newline
usibility optimierungen an der GUI



% !!!! WICHTIG Kommt auf dieser Seite nach ganz unten  für die QUELLEN ----> WICHTIG!!!!


\bibliographystyle{plaindin}
\bibliography{Referenzen}


