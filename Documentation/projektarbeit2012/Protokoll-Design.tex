\label{sec:ProtokolDesign}
\todo{alle bilder mit caption und label!!!}
Dieses Kapitel soll einen Überlick über das eigentliche Design des Content
Relevant-Oriented Protocol (CROP) geben und auf den Aufbau und die
Strukturierung der zu sendenden Nachricht eingehen.\todo{anderen anfang - so
fängt hier jeder absatz an\ldots !! das bitte ähnlich in den anfang des kapitels
einfügen (zu dem todo)}

Zentrale Rolle spielte dabei der Header der Nachricht. Dieser beinhaltet
Informationen, die zum eindeutigen Versenden und Identifizieren der
mitgelieferten Daten notwendig sind. Dazu gehören die Versionsnummer,
die Konfiguration und die Länge der Nachricht und die genauen Adressen
des Senders und Empfängers. Neben dem Header sind die verpackten Daten, der
sogenannte Payload, und ein CRC-Code zur Fehlerüberprüfung der Nachricht
vorhanden. In Abbildung \ref{fig:DatenaufschluesselungMessage} ist der Header
einer Nachricht im Detail dargestellt.

\begin{figure}[H]
	\centering
	\includegraphics[width=\textwidth]{DatenaufschluesselungMessage.png}
	\caption{Datenaufschlüsselung der Nachricht}
	\label{fig:DatenaufschluesselungMessage}
\end{figure}

Die Versionsnummer belegt die ersten vier Bits des Headers. Diese
signalisiert dem Empfänger mit welcher Version des Protokolls die Nachricht
verpackt und versandt wurde. Darauf folgt die Konfiguration. Mit Hilfe dieser,
können Einstellungen vorgenommen werden, welche die Größe der gesamten Nachricht
beeinflussen. Somit ist in speziellen Fällen eine bandbreitenschonende
Übertragung der Nachricht möglich, da keine ungenutzten Informationen oder Bits
vorhanden sind. Für die Konfiguration wurden $12$ Bits reserviert. Die ersten drei Bits
bestimmen das Adressformat. Diese ermöglichen das Aufsetzen des Protokolls auf
bereits bestehenden Standards, wie IPv6 oder des Bundle-Protokolls.
Die verbleibenden neun Bits sind reserviert und dienen zur Erweiterung des
Protokolls für zusätzliche Einstellungsmöglichkeiten. Die Bitvergabe
der Adressen von Sender und Empfänger erfolgt dynamisch abhängig der
Konfiguration.
Dies ist für die Nutzung unterschiedlicher Übertragungsprotkolle notwendig.
Für IPv6 werden $128$ Bit bereitgestellt. Dies sind jeweils $64$ Bit für
die Sender- und Empfängeradresse. Die Länge repräsentiert die Größe der gesamten
Nachricht in Bytes und belegt die nächsten $24$ Bits.
Der vorletzte Bestandteil der Nachricht, der sogenannte Payload, beinhaltet die
eigentlich Daten und besteht aus mehreren Datenblöcken. Am Ende werden noch
die Püfsummenbits zur Fehlererkennung hinzugefügt. Diese sind $16$ Bits lang,
wenn die Gesamtlänge der Nachricht kleiner gleich $2^{16}$ Bytes beträgt.
Andernfalls beträgt die Länge $32$ Bits. Durch diese Unterteilung kann
Overhead vermieden werden und trotzdem ist gewährleistet große Nachrichten
verschicken zu können.

\begin{figure}[H]
	\centering
	\includegraphics[width=\textwidth]{DatenaufschluesselungDB.png}
	\caption{Aufschlüsselung der Datenblöcke}
  \label{fig:DatenaufschluesselungDB}
\end{figure}

Eine schnelle und eindeutige Zuordnung eines einzelnen Datenblockes auf der
Empfängerseite, stand bei der Entwicklung des Headers im Mittelpunkt. Das heißt
neben der Bitvergabe war eine genaue Überlegung über die richtige Reihenfolge
notwendig. Hierzu gibt es die in Abbildung dargestellten drei Ansätze.

\begin{figure}[H]
	\centering
	\includegraphics[width=\textwidth]{DatenblockVarianten.png}
\end{figure}

Ein Datenblock besteht aus den folgenden Teilen: DOID (Data Object
Identification Number), Datentyp, Config, Sequenznummer und der Länge des
Datenblocks. Die DOID repräsentiert die Datei (Bild, Text, Sensorwerte,\ldots)
dem der Datenblock angehört. Das Differenzieren der einzelnen Datenblöcke
untereinander erfolgt mittels der Sequenznummer. Das Einführen des Datentyps im
Header ermöglichte noch mehr unterschiedliche Datenblöcke, da DOID und
Sequenznummer nur einem Datentyp zugeordnet werden.

\todo{siehe kommentar}
% klingt nun besser aber einfach zu lesen ist es noch nicht und
% schwer nachvollziehbar\ldots weiss nichtmal ob da noch kommas fehlen bei den
% schachteldingern ;) formulier das bitte etwas klarer bei variante 3 bring bitte
% noch dne punkt rein das durch die datentyp bits indirekt zusätzliche bits für
% doid zur verfügung stehen (der grosse vorteil der sache)
Ausgegangen wurde anfangs von je acht Bit für Datentyp und Config.
Dabei kam die Frage auf, ob diese wirklich ausreichen, da sehr viele verschiedene Datentypen
und Formate existieren. Infolgedessen wurde, wie in Variante $1$ zu sehen, ein
zusätzliches Byte zur Verfügung gestellt. Aufgeteilt zu zwei Bit für den
Datentyp und sechs Bit für die Config. Dieser Variante lag der Idee zu Grunde,
dass der Datenblock als erstes über die DOID und im Folgenden dem
zugeordneten Datentyp identifiziert wird. Folgen sollte die Config und
anschließend die Sequenznummer. Ein ähnlicher Ablauf galt auch für Möglichkeit
$2$, bei der die restlichen sechs Bit zur Vorreservierung größerer Datentypen
genutzt wurden. Dies war bezüglich des Datentyps und der Menge unterschiedlicher
Datenblöcke die bessere Variante. \todo{müssen wir nochmal genau schauen da bin
ich mir gerade iwi net sicher wie das genau war!!!} Als effektivste Maßnahme
ergab sich am Ende aber nicht aus dem Spendieren eines zusätzlichen Bytes,
sondern die Reihenfolge in der die Bestandteile im Header angeordnet sind. Wie
in Variante $3$ ersichtlich, wurden sechs Bit gespart, die Anordnung aber derart
verändert, dass ein Datenblock zunächst an seinem Datentyp identifiziert wird.
Dies ist sinnvoller, da sofort eine Eingrenzung der möglichen Daten-Objekte
(DOIDs) eintritt. Erst dann folgt die DOID und die Sequenznummer. Die Config
an zweiter Stelle ermöglicht einen schnellen Überblick über eventuelle
Einstellungen des Headers. So existiert neben den Kompressionsmöglichkeiten ein
weiteres Bit in der Config, das einen sogenannten Timestamp aktiviert. Dieser
repräsentiert die Zeit, zu dem die Daten des Datenblocks aufgenommen wurden. Das
Timestamp-Flag in der Config wurde eingeführt, um einen Overhead bei
beispielsweise Texten oder Bildern zu vermeiden. Die Idee besteht darin, dass
auf Grund der geringen Größe einzelner Sensordaten und dem darausfolgenden
Overhead beim Verschicken, mehrere Sensordaten in einem Datenblock
zusammengefasst werden. Da jeder dieser Sensordaten zu einem anderen Zeitpunkt
aufgenommen wurde, ist für diese ein seperater Timestamp notwendig (siehe
Abbildung LETZTE GEILE AUFSPLITTUNG), anders als bei Bildern oder Texten. 

Damit geben die ersten $4$ Bits innerhalb des Datentypes den übergeordneten Typ
(Bild, Text, \ldots) an und die Verbleibenden spezifizieren das genaue Format
(jpg, txt, \ldots). Die Config beträgt $6$ Bits. Dabei sind die vordersten drei
Bits die Kompression des Datenblockheaders, damit kann die Längen der DOID, der
SequenzNummer und der Länge des gesamten Datenblockes varriert und somit
Overhead gespart werden. Das vierte Bit gibt an, ob ein Zeitstempel nach dem
Header von $8$ Byte gesetzt wird.

\includepdf{Datenaufschluesselung.pdf}

Das Bild \ref{fig:beispielJPG} zeigt die beispielhafte Aufsplittung noch einmal
an Hand eines Bildes im Format JPEG. Wie zu erkennen, besteht das Bild aus
mehreren "Contents". Diese wiederum vereinen eine Vielzahl an Pixeln (siehe
"Gesplittetes JPEG" in der Abbildung). Diese Bildaten in Verbindung mit dem
"JPEG-Header" bilden den gesamten Content. Ein Datenblock besteht dann wie oben
beschrieben und im Bild \ref{fig:beispielJPG} (rechter Teil) zu sehen aus diesem
Content und dem dazugehörigen Datenblockheader.

\begin{figure}[H]
	\centering
	\includegraphics[width=\textwidth]{beispielMessage.png}
	\label{fig:beispielJPG}
\end{figure}

Neben der Strukturierung und des Aufbaus der Nachricht selbst ist die
Priorisierung einzelner Datenblöcke sehr wichtig. Diese gibt an in welcher
Reihenfolge die Nachrichten versendet werden. Die Priorisierung erfolgt hierbei
grundlegend in zwei Schritten, wie die Abbildung \ref{fig:priorisierungen}
zeigt.
Zunächst erfolgt eine Vorpriorisierung. Dabei werden relevante Bereiche
selektiert und mit einem Relevanzwert eingestuft. Die
Datei wird daraufhin in einzelne Datenblöcke zerlegt (siehe Abbildung
\ref{fig:priorisierungen} rechts). Diese erhalten abhängig der Wichtigkeit einen
Prioritätswert. Dabei wird ein Verhältnis zwischen dem Datenblock und dem darin
enthaltenem Inhalt berechnet. Nach der Priorisierung erfolgt eine Einsortierung
der Datenblöcke unter Berücksichtigung des Prioritätswertes in die Queue.

\begin{figure}[H]
	\centering
	\includegraphics[width=\textwidth]{Priorisierung.png}
	\label{fig:priorisierungen}
\end{figure}

\todo{relevance value (flag)?? was das}
