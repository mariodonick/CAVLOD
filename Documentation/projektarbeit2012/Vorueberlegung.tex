
% relevanzwerte und priorität erklären (nicht erklären dass die von
%0-100 gehn und prozente darstellen das erfolgt in schnittstellen)

\todo{anfangs bla find ich noch unschön}
Dieses Kapitel befasst sich mit den Vorbetrachtungen zur Projektarbeit. Dabei
geht es um {\"U}berlegungen zur Fehlererkennung und der Handhabung von
Verbindungsabbr{\"u}chen bzw. fehlerhaften {\"U}bertragungen.

\textbf{Time To Live}

Die TTL bezeichnet die Lebensdauer eines Datenpakets und ist dabei von
unterschiedlichen Aspekten abh{\"a}ngig. So kann ein Paket einerseits nach
Ablauf eines Zeitraums verworfen werden oder aber nach einer bestimmten Anzahl
von hops. \todo{vervollständigende frage was sind hops eventl nebensatz
erklärnug} Unter Ber{\"u}cksichtigung einer Interplanetare Kommunikation
w{\"a}re eine TTL Realisierung per Zeitstempel sinnvoll, da hier{\"u}ber
unrealistische {\"U}bertragungszeiten erkannt werden k{\"o}nnen.
\todo{wir hatten glaub ich schon ein erstes konzept/ideen zur umsetzung}

\textbf{Protocol Stack}



\textbf{Error Correction Code}

Zur Fehlererkennung bzw. Korrektur kommt ein CRC-Code zum Einsatz. Dieser kann
innerhalb des Protokolls je nach Paketgr{\"o}{\ss}e auf CRC 16 Bit oder CRC 32
Bit eingestellt werden. Die Pr{\"u}fsumme wird durch einen mathematischen
Algorithmus ermittelt und dann mit dem Paket {\"u}bertragen. Wenn der
Empf{\"a}nger die R{\"u}ckrechnung unter Einbeziehnung der Pr{\"u}fsumme
vornimmt, kann anhand des Ergebnisses ermittelt werden ob das Paket
verf{\"a}lscht wurde. \todo{eventl genauer die beiden CRCs beschreiben? wie die
funktionieren formeln etc?}
