\label{sec:Anwendungsszenarien}

Im Folgenden werden Anwendungsszenarien für das CROP dargestellt. Dabei wird die
Funktionsweise sowohl anhand eines Textes als auch eines Bildes beschrieben und
verglichen. Dies erfolgt von der Eingabe über die Splittung bis hin zum Versand
der Nachricht.

Die Art und Weise wie ein Bild oder ein Text untersucht und bearbeitet wird
unterscheidet sich. Im Rahmen dieser Arbeit wird ein Text als
Anwendungsszenario herangezogen. Grundlage bildet hierbei eine Nachricht,
welche der Nutzer über eine grafische Benutzerberfläche eingibt.
Dabei ist zusätzlich ein manuelles Hervorheben der wichtigen Textpassagen
notwendig. Zur näheren Erläuterung soll folgendes fiktives Testszenario zur
Betrachtung herangezogen werden:

\textit{\glqq Gestern um 8 Uhr Marszeit wurden Hinweise auf mögliches Leben auf
dem Mars entdeckt, wobei ein Mann schwer verletzt wurde! \grqq}

Die höchste Aufmerksamkeit gilt der Tatsache, dass mögliches Leben entdeckt und
dabei ein Mann schwer verletzt wurde. Demnach bekommen diese beiden Abschnitte
die höchste Priorität. Folgen könnte die Uhrzeit, damit der Empfänger zuordnen
kann, wie lange die Verletzung bereits vorliegt. Alle weiteren Wörter stellen
zusätzliche Informationen dar, die für die Vollständigkeit des Textes, aber
nicht zum Verständnis der Situation notwendig sind. Die Abbildung
\ref{fig:prioChatWindow} zeigt die Prioritäten in der ChatGui, auf die im
Kapitel \ref{cap:chatGui} noch näher eingegangen wird.

\begin{figure}[H]
	\centering
	\includegraphics[width=\textwidth]{prioChatWindow.png}
	\label{fig:prioChatWindow}
	\caption{Chat-Fenster mit priorisierter Eingabe}
\end{figure}

Der Text wird jetzt Wort für Wort untersucht und anhand der Priorität sortiert.
Dabei bilden Wortgruppen mit gleicher Relevanz einen Datenblock, dem eine
Sequenznummer zugeordnet wird, wodurch eine eindeutige Identifizierung möglich
ist. Die \gls{DOID} beugt einer Verwechslung anderer Datenblöcke, gleicher
Sequenznummer vor. Weiterhin werden die Position und die Länge der Wortgruppe im
Text abgespeichert. Anhand dieser Informationen ist der Empfänger in der Lage
die ankommende Datenblöcke korrekt zuzuordnen und so den Text wieder herzustellen.
in Abbildung \ref{fig:chatguiexample} ist dargestellt wie einzelne Fragemente
zuerst ankommen und der Text mit jeden weitern Datenpaket vervollständigt wird.

\begin{figure}[H]
	\centering
	\subfigure{\includegraphics[scale=.4]{chatGuiWindow_empfaenger_prio.png}}\hfill
	\subfigure{\includegraphics[scale=.4]{chatGuiWindow_empfaenger_halfPrio.png}}\\
	\subfigure{\includegraphics[scale=.4]{chatGuiWindow_empfaenger_all.png}}
	\label{fig:chatguiexample}
	\caption{Wiederherstellung der Nachricht beim Empfänger}
\end{figure}

Der Ablauf eines Bildes gestaltet sich ähnlich. Die Abbildung
\ref{fig:marsWaterResidue} (a) zeigt das vom Marsrover \textit{Curiosity}
am $28.$ September aufgenommene ausgetrocknete Wasserbett. \todo{quelle} Der im
linken Bild markierte Bereich zeigt, Wissenschaftlern zur Folge, einen vom Wasser
verformten Kiesel. Wie beim Szenario zuvor erfolgt zunächst eine Markierung der wichtigen
Bereiche. Diese sind der Kiesel und der Maßstab, welche in der
Abbildung \ref{fig:marsWaterResidue} (b) dargestellt werden. Bevor das Bild
Schritt für Schritt analysiert wird, muss dieser in Abschnitte eingeteilt
werden, welche später die einzelnen Datenblöcke darstellen.
 
\begin{figure}[H]
	\centering
	\subfigure[Originalbild]{\includegraphics[scale=.4]{marsWaterResidue_links.jpg}}\hfill
	\subfigure[Markierung der Relevanz]
	 {\includegraphics[scale=.4]{marsWaterResidue_mitte.jpg}}\hfill
	\subfigure[Zerlegung des Bildes]
	 {\includegraphics[scale=.4]{marsWaterResidue_rechts.jpg}}
	\label{fig:marsWaterResidue}
	\caption{Priorisierung des Bildes}
\end{figure}

Die weiteren Schritte sind equivalent zu denen des Textes. Die beiden wichtigen
und damit höher priorisierten Bereiche erreichen den Empfänger zuerst.
Das Zusammensetzen des Bildes wird in seiner prinzipiellen Form in
Abbildung \ref{fig:marsWaterResidueEmpfaenger} dargestellt.

\begin{figure}[H]
	\centering
	\subfigure[leeres Bild]
	 {\includegraphics[scale=.4]{marsWaterResidue_empfaenger_links.jpg}}\hfill
	\subfigure[empfangene wichtige Bereiche]
	 {\includegraphics[scale=.4]{marsWaterResidue_empfaenger_mitte.jpg}}\hfill
	\subfigure[wieder zusammengesetztes Bild]
	 {\includegraphics[scale=.4]{marsWaterResidue_empfaenger_rechts.jpg}}
	\label{fig:marsWaterResidueEmpfaenger}
	\caption{Wiederherstellung des Bildes}
\end{figure}