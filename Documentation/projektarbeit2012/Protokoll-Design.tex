\label{sec:ProtokolDesign}

Eine zentrale Rolle bei der Entwicklung des Aufbaus und der Strukturierung
spielte der Nachrichtenheader.
Dieser beinhaltet Informationen, die zum eindeutigen Versenden und Identifizieren der
mitgelieferten Daten notwendig sind. Dazu gehören die Versionsnummer,
die Konfiguration, die Länge der Nachricht und die genauen Adressen
des Senders und Empfängers. Neben dem Header sind die verpackten Daten, der
sogenannte Payload, und ein CRC-Code zur Fehlerüberprüfung der Nachricht
vorhanden. In Abbildung \ref{fig:DatenaufschluesselungMessage} ist der Header
einer Nachricht im Detail dargestellt.

\begin{figure}[H]
	\centering
	\includegraphics[width=\textwidth]{DatenaufschluesselungMessage.pdf}
	\caption{Datenaufschlüsselung der Nachricht}
	\label{fig:DatenaufschluesselungMessage}
\end{figure}

Die Versionsnummer belegt die ersten vier Bits des Headers. Diese
signalisiert dem Empfänger mit welcher Version des Protokolls die Nachricht
verpackt und versandt wurde. Danach folgt die Konfiguration. Mit Hilfe dieser,
können Einstellungen vorgenommen werden, welche die Größe der gesamten Nachricht
beeinflussen. Somit ist in speziellen Fällen eine bandbreitenschonende
Übertragung der Nachricht möglich, da keine ungenutzten Informationen oder Bits
vorhanden sind. Für die Konfiguration wurden $12$ Bits reserviert. Die ersten drei Bits
bestimmen das Adressformat. Dieses ermöglicht das Aufsetzen des Protokolls auf
bereits bestehenden Standards, wie IPv6 oder das Bundle-Protokoll.
Die verbleibenden neun Bits sind für zusätzliche Einstellungsmöglichkeiten zur Erweiterung des
Protokolls reserviert. Die Bitvergabe der Adressen von Sender und Empfänger
erfolgt dynamisch abhängig der Konfiguration.
Dies ist für die Nutzung unterschiedlicher Übertragungsprotkolle notwendig.
Für IPv6 werden $256$ Bit bereitgestellt. Dies sind jeweils $128$ Bit für
die Sender- und Empfängeradresse. Die Länge repräsentiert die Größe der gesamten
Nachricht in Bytes und belegt die nächsten $24$ Bits.
Der vorletzte Bestandteil der Nachricht, der sogenannte Payload, beinhaltet die
eigentlich Daten und besteht aus mehreren Datenblöcken. Zusätzlich werden am
Ende die Prüfsummenbits zur Fehlererkennung hinzugefügt. Diese sind $16$ Bits lang,
wenn die Gesamtlänge der Nachricht kleiner gleich $2^{16}$ Bytes beträgt.
Andernfalls beträgt die Länge $32$ Bits. Durch diese Unterteilung kann
Overhead vermieden werden und trotzdem ist gewährleistet große Nachrichten
verschicken zu können.

\begin{figure}[H]
	\centering
	\includegraphics[width=\textwidth]{DatenaufschluesselungDB.pdf}
	\caption{Aufschlüsselung der Datenblöcke}
  \label{fig:DatenaufschluesselungDB}
\end{figure}

Eine schnelle und eindeutige Zuordnung eines einzelnen Datenblockes auf der
Empfängerseite,  \todo{silbentrennung}stand bei der Entwicklung des Headers im
Mittelpunkt.
Das heißt neben der Bitvergabe war eine genaue Überlegung über die richtige Reihenfolge
notwendig. Hierzu gibt es drei Ansätze, welche in Abbildung
\ref{fig:DatenblockVarianten} dargestellt sind.

\begin{figure}[H]
	\centering
	\includegraphics[width=\textwidth]{DatenblockVarianten.pdf}
	\caption{Ansätze des Datenblockheaders}
  \label{fig:DatenblockVarianten}
\end{figure}

Ein Datenblock besteht aus den folgenden Teilen: DOID \todo{glossar}(Data Object
Identification Number), Datentyp, Konfiguration, Sequenznummer und der Länge des
Datenblocks. Die DOID repräsentiert die Datei (Bild, Text, Sensorwerte,\etc)
dem der Datenblock angehört. Das Differenzieren der einzelnen Datenblöcke
untereinander erfolgt mittels der Sequenznummer. Durch die Einführung eines
Datentyps im Header wird der Bereich der DOID indirekt vergrößert. Dies
ist aufgrund der eindeutigen Zuordnung einer DOID zu einem Datentyp möglich.
\newline
Ausgegangen wurde anfangs von je acht Bit für Datentyp und Konfiguration.
In diesem Zusammenhang wurde hinterfragt, ob die Bitvergabe ausreicht,
weil sehr viele verschiedene Datentypen und Formate existieren. Infolgedessen
wurde, wie in Variante $1$ zu sehen, ein zusätzliches Byte zur Verfügung
gestellt. Aufgeteilt zu zwei Bit für den Datentyp und sechs Bit für die
Konfiguration. Die Idee hinter dieser Variante war, dass der Datenblock als
erstes über die DOID und im Folgenden dem zugeordneten Datentyp identifiziert
wird. Anschließend sollte die Konfiguration und die Sequenznummer folge.
Ein ähnlicher Aufbau gilt ebenfalls für Möglichkeit $2$, bei der die restlichen
sechs Bit zur Vorreservierung größerer Datentypen genutzt wurden. Dies war
bezüglich des Datentyps und der Menge unterschiedlicher Datenblöcke die bessere
Variante. Eine effektivere Maßnahme ergab sich am Ende nicht aus dem
Spendieren eines zusätzlichen Bytes, sondern aus einer anderen Reihenfolge in
der die Bestandteile im Header angeordnet sind. Wie in Variante $3$ ersichtlich,
wurden sechs Bit gespart. Dadurch wird zuerst jeder Datenblock anhand
seines Datentypes identifiziert und anschliessend durch die DOID und die
Sequenznummer spezifiziert. Danach folgt die Konfiguration mit $6$ Bits.
Dabei sind die vordersten drei Bits die Kompression des Datenblockheaders, damit
kann die Längen der DOID, der SequenzNummer und der Länge des gesamten
Datenblockes varriert. Dadurch können auch kompakte Blöcke effizient verschickt
werden.
Das vierte Bit gibt an, ob ein Zeitstempel nach dem Header und vor einem
Datenpacket mit einer Länge von $8$ Byte gesetzt wird.
Dieses Bit wurde in der Konfiguration eingeführt, weil bei einigen Datentypen
nicht zwingend eine Zeitangabe benötigt wird und somit Overhead vermieden
werden kann. Für Daten mit konstanter aber geringer Größe können mehrere Werte
gleichzeitig in einem Datenblock platziert werden. Diese besitzen bei
gesetztem Zeitbit eine eigene Zeitangabe. Die Abbildung
\ref{fig:uebersichtdatenaufschlüsselung} stellt diesen Sachverhalt noch einmal
übersichtlich dar.

\begin{figure}[H]
  \centering
  \includegraphics[width=\textwidth]{Datenaufschluesselung.pdf}
  \caption{Übersicht der Datenaufschlüsselung}
  \label{fig:uebersichtdatenaufschlüsselung}
\end{figure}

Die Abbildung \ref{fig:beispielJPG} zeigt die beispielhafte Aufsplittung
noch einmal an Hand eines Bildes im Format JPEG. Wie zu erkennen, besteht das Bild aus
mehreren "Contents". Diese wiederum vereinen eine Vielzahl an Pixeln (siehe
"Gesplittetes JPEG" in der Abbildung). Diese Bildaten in Verbindung mit dem
"JPEG-Header" bilden den gesamten Content. Ein Datenblock besteht dann wie oben
beschrieben und im Bild \ref{fig:beispielJPG} (rechter Teil) zu sehen aus diesem
Content und dem dazugehörigen Datenblockheader.

\begin{figure}[H]
	\centering
	\includegraphics[width=\textwidth]{beispielMessage.png}
	\caption{muh}
	\label{fig:beispielJPG}
\end{figure}

Neben der Strukturierung und des Aufbaus der Nachricht selbst ist die
Priorisierung einzelner Datenblöcke sehr wichtig. Diese gibt an in welcher
Reihenfolge die Nachrichten versendet werden. Die Priorisierung erfolgt hierbei
grundlegend in zwei Schritten, wie die Abbildung \ref{fig:priorisierungen}
zeigt.
Zunächst erfolgt eine Vorpriorisierung. Dabei werden relevante Bereiche
selektiert und mit einem Relevanzwert eingestuft. Die
Datei wird daraufhin in einzelne Datenblöcke zerlegt (siehe Abbildung
\ref{fig:priorisierungen} rechts). Diese erhalten abhängig der Wichtigkeit einen
Prioritätswert. Dabei wird ein Verhältnis zwischen dem Datenblock und dem darin
enthaltenem Inhalt berechnet. Nach der Priorisierung erfolgt eine Einsortierung
der Datenblöcke unter Berücksichtigung des Prioritätswertes in die Queue.

\begin{figure}[H]
	\centering
	\includegraphics[width=\textwidth]{Priorisierung.png}
	\caption{Datenaufschlüsselung der Nachricht}
	\label{fig:priorisierungen}
\end{figure}

\todo{relevance value (flag)?? was das}
